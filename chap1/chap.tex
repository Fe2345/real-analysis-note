\ifx\allfiles\undefined
\documentclass[12pt, a4paper, oneside, UTF8]{ctexbook}
\def\path{../config}
\usepackage{amsmath}
\usepackage{amsthm}
\usepackage{amssymb}
\usepackage{graphicx}
\usepackage{mathrsfs}
\usepackage{enumitem}
\usepackage{geometry}
\usepackage[colorlinks, linkcolor=black]{hyperref}
\usepackage{stackengine}
\usepackage{yhmath}
\usepackage{extarrows}
\usepackage{unicode-math}
\usepackage{pifont}

\usepackage{fancyhdr}
\usepackage[dvipsnames, svgnames]{xcolor}
\usepackage{listings}

\definecolor{mygreen}{rgb}{0,0.6,0}
\definecolor{mygray}{rgb}{0.5,0.5,0.5}
\definecolor{mymauve}{rgb}{0.58,0,0.82}

\graphicspath{ {figure/},{../figure/}, {config/}, {../config/} }

\linespread{1.6}

\geometry{
    top=25.4mm, 
    bottom=25.4mm, 
    left=20mm, 
    right=20mm, 
    headheight=2.17cm, 
    headsep=4mm, 
    footskip=12mm
}

\setenumerate[1]{itemsep=5pt,partopsep=0pt,parsep=\parskip,topsep=5pt}
\setitemize[1]{itemsep=5pt,partopsep=0pt,parsep=\parskip,topsep=5pt}
\setdescription{itemsep=5pt,partopsep=0pt,parsep=\parskip,topsep=5pt}

\lstset{
    language=Mathematica,
    basicstyle=\tt,
    breaklines=true,
    keywordstyle=\bfseries\color{NavyBlue}, 
    emphstyle=\bfseries\color{Rhodamine},
    commentstyle=\itshape\color{black!50!white}, 
    stringstyle=\bfseries\color{PineGreen!90!black},
    columns=flexible,
    numbers=left,
    numberstyle=\footnotesize,
    frame=tb,
    breakatwhitespace=false,
} 
\usepackage[strict]{changepage} 
\usepackage{framed}
\usepackage{tcolorbox}
\tcbuselibrary{most}

\definecolor{greenshade}{rgb}{0.90,1,0.92}
\definecolor{redshade}{rgb}{1.00,0.88,0.88}
\definecolor{brownshade}{rgb}{0.99,0.95,0.9}
\definecolor{lilacshade}{rgb}{0.95,0.93,0.98}
\definecolor{orangeshade}{rgb}{1.00,0.88,0.82}
\definecolor{lightblueshade}{rgb}{0.8,0.92,1}
\definecolor{purple}{rgb}{0.81,0.85,1}

% #### 将 config.tex 中的定理环境的对应部分替换为如下内容
% 定义单独编号,其他四个共用一个编号计数 这里只列举了五种,其他可类似定义(未定义的使用原来的也可)
\newtcbtheorem[number within=section]{defn}%
{定义}{colback=OliveGreen!10,colframe=Green!70,fonttitle=\bfseries}{def}

\newtcbtheorem[number within=section]{lemma}%
{引理}{colback=Salmon!20,colframe=Salmon!90!Black,fonttitle=\bfseries}{lem}

% 使用另一个计数器 use counter from=lemma
\newtcbtheorem[use counter from=lemma, number within=section]{them}%
{定理}{colback=SeaGreen!10!CornflowerBlue!10,colframe=RoyalPurple!55!Aquamarine!100!,fonttitle=\bfseries}{them}

\newtcbtheorem[use counter from=lemma, number within=section]{criterion}%
{准则}{colback=green!5,colframe=green!35!black,fonttitle=\bfseries}{cri}

\newtcbtheorem[use counter from=lemma, number within=section]{corollary}%
{推论}{colback=Emerald!10,colframe=cyan!40!black,fonttitle=\bfseries}{cor}
% colback=red!5,colframe=red!75!black

% 这个颜色我不喜欢
%\newtcbtheorem[number within=section]{proposition}%
%{命题}{colback=red!5,colframe=red!75!black,fonttitle=\bfseries}{cor}

% .... 命题 例 注 证明 解 使用之前的就可以(全文都是这种框框就很丑了),也可以按照上述定义 ...
\renewenvironment{proof}{\par\textbf{证明:}\;}{\qed\par}
\newenvironment{solution}{\par{\textbf{解:}}\;}{\qed\par}
\newtheorem{proposition}{\indent 命题}[section]
\newtheorem{example}{\indent \color{SeaGreen}{例}}[section] % 绿色文字的 例 ,不需要就去除\color{SeaGreen}{}
\newtheorem*{rmk}{\indent 注}
\usepackage{amssymb}
\setmathfont{LatinModernMath-Regular}
\setmathfont[range=\mathbb]{TeXGyrePagellaMath-Regular}
\def\d{\mathrm{d}}
\def\R{\mathbb{R}}
\def\C{\mathbb{C}}
\def\Q{\mathbb{Q}}
\def\N{\mathbb{N}}
\newcommand{\bs}[1]{\boldsymbol{#1}}
\newcommand{\ora}[1]{\overrightarrow{#1}}
\newcommand{\myspace}[1]{\par\vspace{#1\baselineskip}}
\newcommand{\xrowht}[2][0]{\addstackgap[.5\dimexpr#2\relax]{\vphantom{#1}}}
\newenvironment{ca}[1][1]{\linespread{#1} \selectfont \begin{cases}}{\end{cases}}
\newenvironment{vx}[1][1]{\linespread{#1} \selectfont \begin{vmatrix}}{\end{vmatrix}}
\newcommand{\tabincell}[2]{\begin{tabular}{@{}#1@{}}#2\end{tabular}}
\newcommand{\pll}{\kern 0.56em/\kern -0.8em /\kern 0.56em}
\newcommand{\dive}[1][F]{\mathrm{div}\;\bs{#1}}
\newcommand{\rotn}[1][A]{\mathrm{rot}\;\bs{#1}}
\usepackage{xeCJK}
\setCJKmainfont{SimSun}[BoldFont={SimHei}, ItalicFont={KaiTi}] % 设置中文支持

\newcommand{\point}[1]{\item {#1}}
\newenvironment{para}[1]{%
\ifcase#1\relax
\begin{enumerate}[label=\arabic*.] % 1.2.3.
\or
\begin{enumerate}[label=\textcircled{\arabic*}] % ①②③
\or
\begin{enumerate}[label=(\roman*)] % (i)(ii)(iii)
\else
\begin{enumerate}[label=\arabic*.] % 默认格式
\fi
}{
\end{enumerate}
}

\def\myIndex{0}
% \input{\path/cover_package_\myIndex.tex}

\def\myTitle{实分析笔记}
\def\myAuthor{Zhang Liang}
\def\myDateCover{\today}
\def\myDateForeword{\today}
\def\myForeword{前言标题}
\def\myForewordText{
    前言内容
}
\def\mySubheading{副标题}


\begin{document}
% \input{\path/cover_text_\myIndex.tex}

\newpage
\thispagestyle{empty}
\begin{center}
    \Huge\textbf{\myForeword}
\end{center}
\myForewordText
\begin{flushright}
    \begin{tabular}{c}
        \myDateForeword
    \end{tabular}
\end{flushright}

\newpage
\pagestyle{plain}
\setcounter{page}{1}
\pagenumbering{Roman}
\tableofcontents

\newpage
\pagenumbering{arabic}
\setcounter{chapter}{0}
\setcounter{page}{0}

\pagestyle{fancy}
\fancyfoot[C]{\thepage}
\renewcommand{\headrulewidth}{0.4pt}
\renewcommand{\footrulewidth}{0pt}








\else
\fi
%标题
\chapter{测度论}
	实分析中我们希望解决一个问题:如何扩展黎曼积分的定义,使得一些连续性不太好的函数也可积?
	
	黎曼积分是基于“分割”的,强调的是一个区间上的积分;我们尝试将其扩展到具有“长度”的集合上。
	
	我们先讨论如何建立“长度”的精确定义
	\section{外测度}
	我们提出一些最为基础的想法:如果可以用一系列“盒子”覆盖一个集合,我们可以主观地认为,这个集合的体积小于这个“复合盒子”的大小。
	如果取这些大小的下确界,我们就可以将其视为这个集合的体积。因此我们提出外测度的概念。
		\subsection{区间的体积}
			首先我们先考察区间的体积如何定义。我们将在后续看到,为何不在此直接定义出区间的外测度。
			
			先给出区间(矩形)、正方体的定义
			\begin{defn}{区间、正方体}{}
				集合$[a,b]=[a_1,b_1] \times \cdots [a_n,b_n] \subset \R^n,a_i \leqslant b_i,a=(a_1,\cdots,a_n)\in\R^n,b=(b_1,\cdots,b_n)\in\R^n$称为以$a,b$为端点的一个闭区间;
				
				集合$(a,b)=(a_1,b_1) \times \cdots (a_n,b_n) \subset \R^n,a=(a_1,\cdots,a_n)\in\R^n,b=(b_1,\cdots,b_n)\in\R^n$称为以$a,b$为端点的一个闭区间;
				
				如果$[a,b]$有$\forall i \in \{1,\cdots,n\},b_i-a_i$为定值,那么称$[a,b]$为一个正方体;
				
				如果$(a,b)$有$\forall i \in \{1,\cdots,n\},b_i-a_i$为定值,那么称$(a,b)$为一个正方体;
				
				闭区间和开区间统称为区间(也称为矩形)。
				
				并规定:如果两个区间$A,B$仅在其边界$\partial A,\partial B$上有公共点,即$(A \cap B )\subseteq (\partial A \cap \partial B)$,那么称这两个区间几乎不相交。
			\end{defn}
			接下来,我们给出闭区间的体积的定义
			\begin{defn}{闭区间的体积}
				我们定义闭区间$R = [a,b]$的体积
				
				$|R| = \prod\limits_{i=1}^{n} (b_i-a_i)$
			\end{defn}
			\subsection{外测度的定义}
			于是可以给出外侧度的定义:
			\begin{defn}{外侧度}{}
				设集合$E \subseteq \R^n$,集合族$\{Q_i\}$覆盖$E$,即$E \subseteq \bigcup\limits_{i=1}^{\infty} Q_i$
				
				我们定义:
				
				$m_*(E) = inf \sum\limits_{i=1}^{\infty} |Q_i|$,称为$E$的外侧度
			\end{defn}
			在考察外测度性质前,我们先证明几个引理,并考察几个比较平凡的集合的外测度。
			\begin{lemma}{闭区间体积的有限可加性}{}
				设集合$R$是有限个几乎不相交的闭区间的并,即$R = \bigcup\limits_{i=1}^{n} R_i$
				
				那么有:$|R| = \sum\limits_{i=1}^{n} |R_i|$
			\end{lemma}
			\begin{proof}
				延展这些区间$R_i$,一定会产生一系列网格,以及${1,\cdots,M}$的一个分割$J_1,\cdots,J_N$,使得
				
				$R = \bigcup\limits_{j=1}^{M} \tilde{R_j},R_k = \bigcup\limits_{j \in J_k} \tilde{R_j}$,并且$\tilde{R_j}$几乎不相交
				
				实际上,数字$1,\cdots,M$正标明了这些方格中包含$R$的那些方格,$J_k$正是包含了$R_k$的那些序号的集合.
				
				显然,一定有$|R|=\sum\limits_{j=1}^{M} |\tilde{R_j}|$,于是有
				
				$|R| = \sum\limits_{j=1}^{M} |\tilde{R_j}| = \sum\limits_{k=1}^{n} \left(\sum\limits_{j \in J_k} |\tilde{R_j}|\right)$
				
				$=\sum\limits_{k=1}^{n} |R_k|$,因为每一个小区间$R_k$也是一系列区间的并
			\end{proof}
			那么我们如果取消上面的几乎不相交的条件,那么重复这个过程就可以得到这个结论:
			\begin{corollary}{}{}
				设集合$R$包含于有限个闭区间的并,即$R  \subseteq \bigcup\limits_{i=1}^{n} R_i$
				
				那么有:$|R| \leqslant \sum\limits_{i=1}^{n} |R_i|$
			\end{corollary}
			我们接下来转向不规则集合的外测度
			\begin{them}{}{}
				$\R$的任意一个开子集$O$都可以唯一地写成可数个不相交开区间的并
			\end{them}
			\begin{proof}
				取$x \in O$,接下来,取
				
				$a_x = inf \{a| a<x,(a,x) \subseteq O\},b_x = sup \{b|b>x,(x,b) \subseteq O\}$
				
				于是,$I_x = (a_x,b_x)$是包含$x$的最大的$O$的子区间,因此有
				
				$O = \bigcup\limits_{x \in O} I_x$
				
				我们接下来证明唯一性。只需要证明,不同的这些$I_x$不相交即可。
				
				取$I_x,I_y$,假设$I_x \cap I_y \neq \varnothing$,那么又因为$I_x,I_y$分别是包含$x,y$的最大子区间,于是一定有
				
				$(I_x \cap I_y) \subseteq I_x,(I_y \cap I_x) \subseteq I_y \Leftrightarrow I_x = I_y$,这样这一部分就证明完毕了。
				
				接下来证明这些区间的数量可数。由有理数的性质我们知道,每个区间至少包含一个有理数,又因为$\Q$可数,这些区间的数量自然也是可数的。
			\end{proof}
			我们尝试将这个结果推广到$\R^d$
			\begin{them}{}{}
				$R^d$中的每个开子集$O$都可以唯一地写成可数个几乎不相交的闭区间的并
			\end{them}
			\begin{proof}
				我们将所有坐标为整数的点,相互连接,形成一个所有线段都平行于坐标轴的一系列网格。
				
				随后,我们按以下步骤重复操作:
				
				首先选取完全包含于$O$的网格将其接受,完全不包含的则排除,剩余的保留下来。
				
				接下来,将剩余的网格,全部按坐标的中心分割为$2^d$个相同的小区间。
				
				最后,我们把这些小区间组成的一系列网格,重复上面的两步。
				
				这个过程一定会构造一个$O = \bigcup\limits_{s \in A} R_s$,其中$R_x$是这些小闭区间。因为,每一个$x \in O$,都一定能找到一个小正方体包含它。
				
				显然,$A$一定至多可数,因为整个过程在可数步内完成,这意味着,一定有$\text{Card}A \leqslant \text{Card}\N^d$。
				
				唯一性由构造方式是显然的。
			\end{proof}
			在考虑这些引理后,我们开始计算一些集合的测度。
			\begin{example}
				单点的外测度$m_*(a)=0,a\in \R^d$
			\end{example}
			\begin{proof}
				其实,单个点也是一个区间,只不过它的任意方向的上下界都相等。那么,这个点就覆盖自己。
				
				单点的体积$|a|=0$,因为外测度非负,所以只能有$m_*(a)=0$
			\end{proof}
			\begin{example}
				闭正方体的外测度就是它的体积,即$m_*(R) = |R|$
			\end{example}
			值得注意的是,这个命题不能因为闭正方体覆盖自己而直接得证,因为与前面的单点不同,我们仅由这一点不能断定没有其他覆盖的体积小于$|R|$
			\begin{proof}
				首先,闭正方体自己覆盖自己,所以一定有$m_*(R) \leqslant |R|$,我们只需证明相反的不等式。
				
				取$R$的一个覆盖$R \subseteq \bigcup\limits_{i=1}^{\infty} R_i$
				
				取$\varepsilon >0$,于是对于每一个$R_i$,一定可以找到一个开区间$S_i,R_i \subseteq S_i,|S_i| \leqslant (1+\varepsilon) |R_i|$
				
				于是,$\{S_i\}$也构成了$R$的一个覆盖$R\subseteq \bigcup\limits_{i=1}^{\infty} S_i$。
				
				因为$R$是闭正方体,因此是紧集,于是一定能取一个有限覆盖$R \subseteq \bigcup\limits_{i=1}^{n} S_i$。
				
				取$S_i$的闭包,那么也有$R \subseteq \bigcup\limits_{i=1}^{n} \={S_i}$,那么利用前面的引理1.1.2得到:
				
				$|R| \leqslant \sum\limits_{i=1}^{n} |\={S_i}|=\sum\limits_{i=1}^{n} |S_i|$
				
				于是有$|R| \leqslant (1+\varepsilon)\sum\limits_{i=1}^{n} |R_i| \leqslant (1+\varepsilon)\sum\limits_{i=1}^{\infty} |R_i|$
				
				但是$\bigcup\limits_{i=1}^{n} |R_i|$覆盖$R$,$\varepsilon$也是随意选区的,于是有$|R| \leqslant m_*(R)$。那么命题得证。
			\end{proof}
			\begin{example}
				开正方体的外测度$m_*(R)=|\={R}|$
			\end{example}
			\begin{proof}
				首先,$\={R}$覆盖$R$,所以必须有$m_*(R) \leqslant |\={R}|$
				
				我们证明相反的不等式。取一个闭正方体$R_0 \subseteq R$,于是一定有$m_*(R_0) \leqslant m_*(R)$,但是$\={R}$也覆盖$R_0$,于是$|R_0| \leqslant m_*(R)$
				
				但是,$|R_0|$可以无限接近于$R$,于是一定有$m_*(R) \leqslant |R|$,那么命题得证
			\end{proof}
			\begin{example}
				区间的外测度$m_*(R) = |\={R}|$
			\end{example}
			\begin{proof}
				首先,显然有$m_*(R) \leqslant |\={R}|$
				
				我们接下来证明相反的不等式。作一个$\R^d$中边长为$\frac{1}{k}$的网络,我们取那些包含在$R$中的网格,形成集合$J$,并将那些与$R$相交的网格形成集合$K$
				
				那么$R \subseteq \bigcup\limits_{Q \in (J \cup K)} Q$,并且有$\sum\limits_{Q \in J} |Q| \leqslant |R|$ 
				
				我们计算$K$中包含了$R$中某一部分的网格体积之和,显然有$O(k^{d-1})$个这样的网格,每一个网格的体积为$O(k^d)$,
				
				那么总体积$\sum\limits_{Q \in J} |Q| = O(\frac{1}{k})$
				
				于是$\sum\limits_{Q \in (J \cup K)} |Q| \leqslant |R| +O(\frac{1}{k})$,取$k \rightarrow \infty$即有$m_*(R) \leqslant |\={R}|$
			\end{proof}
			\begin{example}
				$m_*(\R^d) = +\infty$
			\end{example}
			\begin{proof}
				显然,一个覆盖如果覆盖了$\R^d$,那么它一定也覆盖了全部的区间,因为任意一个区间都是$\R^d$的子集;
				
				但是,区间的体积可以取得任意大,所以只能有$m_*(\R^d)=+\infty$
			\end{proof}
			\begin{example}
				我们如下定义康托尔集:
				
				我们先定义$C_0 = [0,1]$
				
				随后定义:$C_{n+1}$是$C_{n}$中将每一个$I_x$平分,并去除中间一段的开区间得到的集合。其中$I_x$是包含$x$的最大子区间
				
				我们定义:$C = \bigcap\limits_{i=0}^{\infty} C_i$,称为康托尔集
				
				$m_*(C)=0$
			\end{example}
			\begin{proof}
				由康托尔集的定义可知:$m_*(C_{n+1})=\frac{2}{3} m_*(C_{n})$
				
				所以一定有$m_*(C)=0$
			\end{proof}
		\subsection{外测度的性质}
		\begin{para}{0}
			\point{单调性}
				\begin{proposition}
					如果$E \subseteq F$,那么$m_*(E) \leqslant m_*(F)$
				\end{proposition}
				\begin{proof}
					$F$的覆盖一定也是$E$的覆盖,所以这个命题显然成立
				\end{proof}
			\point{可数可加性}
				\begin{proposition}
					如果$E \subseteq \bigcup\limits_{i=1}^{\infty} E_i$,
					
					那么有$m_*(E) \leqslant \sum\limits_{i=1}^{\infty} m_*(E_i)$
				\end{proposition}
				\begin{proof}
					首先,如果有一个$m_*(E_j) = \infty$,那么命题显然成立。我们接下来对有限外测度情形考虑。
					
					那么,$\forall \varepsilon > 0$,对于任意一个$E_i$,一定能找到一个闭区间覆盖$E_i \subseteq \bigcup\limits_{j=1}^{\infty} R_{i,j}$,其中$R_{i,j}$是闭区间,并且有$\sum\limits_{j=1}^{\infty} |R_{i,j}| \leqslant m_*(E_i) + \frac{\varepsilon}{2^i}$
					
					那么,$E \subseteq \bigcup\limits_{i=1,j=1}^{\infty} R_{i,j}$是$E$的一个覆盖。
					
					于是有$m_*(E) \leqslant \sum\limits_{i,j=1}^{\infty} |R_{i,j}| = \sum\limits_{i=1}^{\infty} \left(\sum\limits_{j=1}^{\infty} |R_{i,j}|\right)$
					
					$\leqslant \sum\limits_{i=1}^{\infty} \left(m_*(E_i) + \frac{\varepsilon}{2^i}\right) = \sum\limits_{i=1}^{\infty} m_*(E_i) + \varepsilon$。
					
					于是命题得证。
				\end{proof}
			\point{}
				\begin{proposition}
					设$E \subseteq R^d$,那么有$m_*(E) = \text{inf }m_*(O)$,其中$O$是开集。
				\end{proposition}
				\begin{proof}
					首先,$m_*(E) \leqslant \text{inf }m_*(O)$显然成立。我们接下来证明相反的不等式
					
					$\forall \varepsilon > 0$,取一个闭区间覆盖$E \subseteq \bigcup\limits_{i=1}^{\infty} R_i$,并且$\sum\limits_{i=1}^{\infty} |R_i| \leqslant m_*(E) + \frac{\varepsilon}{2}$
					
					对于每一个$R_i$,可以选取一个开区间$R_i^0$,使得$R_i \subseteq R_i^0,|R_i^0| \leqslant |R_i| + \frac{\varepsilon}{2^{i+1}}$
					
					那么按照开集的性质,$O = \bigcup\limits_{i=1}^{\infty} R_i^0$也是开集。此时我们利用前面的命题得到:
					
					$m_*(O) \leqslant \sum\limits_{i=1}^{\infty} m_*(R_i^0) = \sum\limits_{i=1}^{\infty} |R_i^0|$
					
					$\leqslant \sum\limits_{i=1}^{\infty} \left(|R_i| + \frac{\varepsilon}{2^{i+1}}\right)$
					
					$=\sum\limits_{i=1}^{\infty} |R_i| + \frac{\varepsilon}{2} \leqslant m_*(E) + \varepsilon$
					
					于是我们证明了相反的不等式$m_*(O) \leqslant m_*(E)$,那么命题得证。
				\end{proof}
		\end{para}
	\section{可测集和Lebesgue测度}
	\section{可测函数}
\ifx\allfiles\undefined
\end{document}
\fi